\clearpage
\section{Conclusione}
Al termine di questa relazione possiamo dire di aver messo in luce le caratteristiche fondamentali del campione in oggetto: abbiamo calcolato e confrontato i più rappresentativi indici per ogni voce in capitolo, abbiamo mostrato graficamente la distribuzione dei dati e infine abbiamo eseguito due test che hanno rivelato dipendenza tra alcuni dei parametri considerati.
Possiamo dedurre, specialmente osservando i grafici, che l’andamento mostrato dai vari sottocampioni presi in esame rispetta l’assunzione di normalità, prevista dalla teoria probabilistica, che riguarda l’adattamento di un grande numero di variabili aleatorie indipendenti a una distribuzione normale standard.  
Si può anche desumere una certa uniformità nei risultati confrontando i diversi corsi di studio, che presentano medie e varianze abbastanza simili fra loro, come è possibile osservare nella tabella generale.
Una nota finale sui test statistici: anche qui abbiamo osservato, come ci si può logicamente aspettare, che un buon senso della logica va di pari passo con una valida mente matematica, e che un risultato positivo alla maturità è un buon lasciapassare negli anni a venire!
\clearpage

\section{Bibliografia}
Per questa relazione di laboratorio abbiamo consultato i seguenti libri di testo:
\begin{itemize}
  \item "Probabilità e statistica per l’Ingegneria e le Scienze", \textit{William Navidi}, McGrow-Hill."
  \item "Analisi statistica dei dati con R", \textit{Franco Crivellari}, Apogeo.
  \item "ggplot2 - Elegant Graphics for Data Analysis", \textit{Hadley Wickham}, Springer.
  \item "Probabilità e statistica per l’Ingegneria e le Scienze", \textit{Sheldon M. Ross}, Apogeo.
\end{itemize}

\section{Sitografia}

Il dataset utilizzato è disponibile a questo indirizzo:
\url{https://raw.github.com/Arguggi/Relazione-Stat/master/dati.csv}\\

\noindent Il codice R usato per analizzare i dati e per creare i grafici è disponibile a questo indirizzo: \\
\url{https://github.com/Arguggi/Relazione-Stat/tree/master/R}
