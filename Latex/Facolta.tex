\section{Divisione Per Facoltà}

E’ inoltre statisticamente interessante riportare i dati relativi a dei sottocampioni logici del campione iniziale per poi svolgere un’indagine tra le differenze sostanziali che potrebbero sorgere dalla suddivisione in sottogruppi per fattori quali, nel nostro caso, il corso scelto da ogni studente per il suo futuro piano degli studi.  In particolare riproponiamo qui di seguito i dati statistici salienti, come sopra, di ogni suddivisione per corso, confrontandoli in particolare tramite boxplot e istogrammi. Sarà interessante notare che non vi saranno grandi differenze con i dati statistici sopra riportati in quanto la maggior parte dei sottocampioni risulta numerosa a sufficienza e indipendente, e, difatti, non si han validi motivi di pensare che la scelta di un corso di studio possa determinare la preparazione di alunno e  viceversa, trattandosi in questo caso di discipline piuttosto affini. Ovviamente, è lecito pensare che studenti che si sentano più ferrati in discipline scientifiche preferiscano seguire corsi scientifici o che includano più matematica rispetto ad altri, quindi sarà posto un occhio di riguardo per discipline del test quali matematica o logica e i corsi anche se difficilmente tali dati, trattandosi ancora di corsi molto affini, potranno risultare talmente differenti da far ipotizzare una qualche dipendenza sostanziale tra i vari risultati nel test, o nel diploma e il corso che preferibilmente un alunno vorrebbe seguire.  Nel caso di ambiguità si procederà dunque a un test di indipendenza tra i corsi e 

Per ogni corso riportiamo quindi media campionaria, varianza campionaria, deviazione standard campionaria, boxplot e istogramma.
\clearpage
\thispagestyle{empty} %Inizio Stat. e Informatica
\newgeometry{top=1cm,bottom=1cm}
\subsection{Stat. e informatica gest. Imprese}

È il corso che implica più materie matematiche e logiche  nel piano di studi.
Calcoliamo le caratteristiche salienti.
Tramite le formule sopra citate calcoliamo media campionaria, varianza campionaria e deviazione standard campionaria.

\begin{center}
\begin{tabular}{|c|c|c|c|}
  \hline
  Sezione & \(x_{n}^{*}\) & \(s_n^2\) & \(s_n\) \\
  \hline
  Voto Finale & 65.02 & 102.45 & 10.12 \\
  \hline
\end{tabular}
\end{center}

\subsubsection{Istogramma}
\begin{figure}[!h]
  \makebox[\textwidth]{%
  \includegraphics[scale=0.51]{STAT._E_INFORMATICA_GEST._IMPRESE-Totale-ist}}
  \caption{Stat. e informatica gest. Imprese}
\end{figure}

\subsubsection{Boxplot}
\begin{figure}[!h]
  \centering
  \includegraphics[scale=0.51]{STAT._E_INFORMATICA_GEST._IMPRESE-Totale-box}
  \caption{Stat. e informatica gest. Imprese}
\end{figure}
\restoregeometry
\clearpage

\thispagestyle{empty} %Inizio Stat. e Informatica
\newgeometry{top=1cm,bottom=1cm}
\subsection{Commercio Estero}

\begin{center}
\begin{tabular}{|c|c|c|c|}
  \hline
  Sezione & \(x_{n}^{*}\) & \(s_n^2\) & \(s_n\) \\
  \hline
  Voto Finale & 70.36 & 121.08 & 11.00 \\
  \hline
\end{tabular}
\end{center}

\subsubsection{Istogramma}
\begin{figure}[!h]
  \makebox[\textwidth]{%
  \includegraphics[scale=0.57]{COMMERCIO_ESTERO-Totale-ist}}
  \caption{Commercio Estero}
\end{figure}

\subsubsection{Boxplot}
\begin{figure}[!h]
  \centering
  \includegraphics[scale=0.57]{COMMERCIO_ESTERO-Totale-box}
  \caption{Commercio Estero}
\end{figure}
\restoregeometry
\clearpage

\thispagestyle{empty} %Inizio Stat. e Informatica
\newgeometry{top=1cm,bottom=1cm}
\subsection{Economia Aziendale}

\begin{center}
\begin{tabular}{|c|c|c|c|}
  \hline
  Sezione & \(x_{n}^{*}\) & \(s_n^2\) & \(s_n\) \\
  \hline
  Voto Finale & 67.03 & 115.15 & 10.73 \\
  \hline
\end{tabular}
\end{center}

\subsubsection{Istogramma}
\begin{figure}[!h]
  \makebox[\textwidth]{%
  \includegraphics[scale=0.57]{ECONOMIA_AZIENDALE-Totale-ist}}
  \caption{Economia Aziendale}
\end{figure}

\subsubsection{Boxplot}
\begin{figure}[!h]
  \centering
  \includegraphics[scale=0.57]{ECONOMIA_AZIENDALE-Totale-box}
  \caption{Economia Aziendale}
\end{figure}
\restoregeometry
\clearpage

\thispagestyle{empty} %Inizio Stat. e Informatica
\newgeometry{top=1cm,bottom=1cm}
\subsection{Marketing e gest. Imprese}

\begin{center}
\begin{tabular}{|c|c|c|c|}
  \hline
  Sezione & \(x_{n}^{*}\) & \(s_n^2\) & \(s_n\) \\
  \hline
  Voto Finale & 65.26 & 107.37 & 10.36 \\
  \hline
\end{tabular}
\end{center}

\subsubsection{Istogramma}
\begin{figure}[!h]
  \makebox[\textwidth]{%
  \includegraphics[scale=0.57]{MARKETING_E_GEST._IMPRESE-Totale-ist}}
  \caption{Marketing e gest. Imprese}
\end{figure}

\subsubsection{Boxplot}
\begin{figure}[!h]
  \centering
  \includegraphics[scale=0.57]{MARKETING_E_GEST._IMPRESE-Totale-box}
  \caption{Marketing e gest. Imprese}
\end{figure}
\restoregeometry
\clearpage

\thispagestyle{empty} %Inizio Stat. e Informatica
\newgeometry{top=1cm,bottom=1cm}
\subsection{Economia Gest. Serv. Turistici}

\begin{center}
\begin{tabular}{|c|c|c|c|}
  \hline
  Sezione & \(x_{n}^{*}\) & \(s_n^2\) & \(s_n\) \\
  \hline
  Voto Finale & 67.28 & 120.36 & 10.97 \\
  \hline
\end{tabular}
\end{center}

\subsubsection{Istogramma}
\begin{figure}[!h]
  \makebox[\textwidth]{%
  \includegraphics[scale=0.57]{ECONOMIA_GEST._SERV._TURISTICI-Totale-ist}}
  \caption{Economia Gest. Serv. Turistici}
\end{figure}

\subsubsection{Boxplot}
\begin{figure}[!h]
  \centering
  \includegraphics[scale=0.57]{ECONOMIA_GEST._SERV._TURISTICI-Totale-box}
  \caption{Economia Gest. Serv. Turistici}
\end{figure}
\restoregeometry
\clearpage

