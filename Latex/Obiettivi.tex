\section{Obiettivi}

Elaborazione statistica dei dati riguardanti i risultati dei test di ammissione,
in particolare studio delle distribuzioni dei dati intorno al loro valor medio
(mediante istogrammi e boxplot); calcolo della media campionaria dei dati,
suddivisi per categoria oppure analizzati come voto globale; calcolo della varianza
e della deviazione standard delle suddette grandezze; visualizzazione grafica
dei risultati ottenuti. Abbiamo poi eseguito dei test statistici per verificare
alcune correlazioni fra le grandezze: collegamento fra voto di diploma e risultato
della prova; correlazione tra le attitudini in logica e in matematica; verifica
di dipendenza/indipendenza fra esito del test e corso di studio prescelto (al
momento ancora non effettuato).

