\section{Statistica Inferenziale}
Abbiamo deciso inoltre di eseguire dei test statistici per verificare la veridicità o meno di alcune ipotesi che ci sembrava sensato fare: in particolare, abbiamo voluto verificare se ci fosse dipendenza tra il voto del diploma e il risultato del test e una eventuale correlazione tra il risultato nella parte di matematica e quello della parte di logica; in entrambi i casi abbiamo considerato l’assenza di correlazione tra i risultati come la nostra ipotesi di partenza.

\subsection{Test del \(\chi^2\) di Indipendenza}

\[
X^2 = \sum_{i=1}^g\sum_{j=1}^k\frac{(n_{ij} - E_{ij})^2}{E_{ij}}
\]
\subsubsection{\(\chi^2\) Test-Maturità}

\begin{center}
  \makebox[\textwidth]{
    \begin{tabular}{|c|c|c|}
    \hline
     &  Test $<$ 65 & Test $>$ 65 \\ 
    \hline
    Voto $<$ 69.5 & 362 &  43 \\ 
    70 $<$ Voto $<$ 80 & 373 &  73 \\ 
    80 $<$ Voto $<$ 90 & 270 &  72 \\ 
    Voto $>$ 90 & 258 & 125 \\ 
   \hline
   \end{tabular}
  }
\end{center}


\begin{center}
  \makebox[\textwidth]{
    \begin{tabular}{|c|c|c|}
    \hline
    \multicolumn{3}{ |c| }{Test Usato \cellcolor[gray]{0.6} } \\
    \hline
    \multicolumn{3}{ |c| }{Pearson's Chi-squared test with Yates' continuity correction} \\
    \hline
    \(\chi^2\) \cellcolor[gray]{0.6}& Gradi di Libertà \cellcolor[gray]{0.6}& p-value \cellcolor[gray]{0.6}\\
    \hline
    64.74 & 3 & $<$ 5.64*\(10^{-14}\) \\
    \hline
    \end{tabular}
  }
\end{center}

In questo caso vediamo che i risultati ottenuti consentono di rifiutare l’ipotesi nulla, evidenziando quindi una relazione di dipendenza tra i risultati in questione, avendo come conseguenza che un punteggio alto all’uscita dalle scuole comporterà con buona probabilità un punteggio alto nel test di ammissione, e di conseguenza un voto totale migliore.

\subsubsection{\(\chi^2\) Logica-Matematica}

\begin{center}
  \makebox[\textwidth]{
    \begin{tabular}{|c|c|c|}
    \hline
    & Matematica $<$ 20 & Matematica $>$ 20 \\
    \hline
    Logica $<$ 20 & 660 & 274 \\
    \hline
    Logica $>$ 20 & 205 & 437 \\
    \hline
    \end{tabular}
  }
\end{center}

Per quanto riguarda l’analisi di una possibile correlazione tra i punteggi di logica e matematica, siamo anche qui partiti dall’ipotesi nulla di assenza di connessione, conseguendo questi risultati:

\begin{center}
  \makebox[\textwidth]{
    \begin{tabular}{|c|c|c|}
    \hline
    \multicolumn{3}{ |c| }{Test Usato\cellcolor[gray]{0.6}} \\
    \hline
    \multicolumn{3}{ |c| }{Pearson's Chi-squared test with Yates' continuity correction} \\
    \hline
    \(\chi^2\) \cellcolor[gray]{0.6}& Gradi di Libertà \cellcolor[gray]{0.6}& p-value \cellcolor[gray]{0.6}\\
    \hline
    228.95 & 1 & $<$ 2.2*\(10^{-16}\) \\
    \hline
    \end{tabular}
  }
\end{center}

Anche in questo caso i valori non ci permettono di confermare l’ipotesi iniziale, inducendoci a pensare ad un collegamento nei punteggi per queste due sezioni del test; possiamo quindi affermare che un punteggio alto in logica quasi sempre implica un punteggio elevato anche per quanto riguarda la sezione di matematica.
